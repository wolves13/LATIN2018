\documentclass{llncs}

\title{Bit bifurcation by cotranscriptional folding\thanks{This work is in part supported by JST Program to Disseminate Tenure Tracking System, MEXT, Japan, No. 6F36 and JSPS Grant-in-Aid for Young Scientists (A) No. 16H05854 to S.~S.}}
\author{Yusei Masuda \and Shinnosuke Seki\thanks{Corresponding author} \and Yuki Ubukata}
\institute{
Department of Computer and Network Engineering, 
The University of Electro-Communications, 
1-5-1, Chofugaoka, Chofu, Tokyo, 1828585, Japan \email{s.seki@uec.ac.jp}
}

\begin{document}

\maketitle

\begin{abstract}
We demonstrate cotranscriptional folding of a finite part of the Heighway dragon, a fractal also-known as the paperfolding sequence $P = LLRLLRRL \cdots$ by an oritatami system. 
The $i$-th element of $P$ can be obtained by feeding $i$ in binary to a 4-state deterministic finite automaton with output (DFAO). 
We implement this DFAO and a bit-string bifurcator as modules of oritatami system. 
Combining them with a known binary counter module yields the target oritatami system. 
%
\begin{keywords}
Heighway dragon, 
Fractal, 
Cotranscriptional folding, 
Oritatami system, 
Automatic sequence, 
Bitstring bifurcation
\end{keywords}
\end{abstract}

%-------------------------------------------------------------------------------------
	\section{Introduction}
%-------------------------------------------------------------------------------------

Molecular shape self-assembly is an engineering technology that provides a design of molecules that organize into a complex target shape due to some simple foundational phenomenon in which molecules interact with each other locally without or with little external control. 
The foundational phenomenon studied most intensively so far is the coalescence of DNA rectangular tiles via their sticky ends (see, e.g., \cite{EvansPhD,Patitz2016,RoPaWi2004}). 
Geary, Rothemund, and Andersen have recently shed light on self-assembly capability of another phenomenon called \textit{RNA cotranscritpional folding} \cite{GeRoAn2014}. 
An RNA sequence is synthesized (\textit{transcribed}) sequentially out of its template (complementary) DNA sequence by an RNA polymerase enzyme. 
The sequence thus synthesized is called a transcript. 
While being transcribed, the transcript folds into an intricate tertiary structure. 
This is the cotranscriptional folding. 
In \cite{GeRoAn2014}, the authors demonstrated an architecture of DNA template sequence whose RNA transcript folds into an RNA rectangular tile as intended in a laboratory. 
\textit{Oritatami system} is a novel mathematical model of RNA cotranscriptional folding \cite{GeMeScSe2016}. 
Computational aspects of RNA cotranscriptional folding such as counting in binary and Turing universality have been studied  \cite{GeMeScSe2017,GeMeScSe2016,HaKiOtSe2016,AnttiPhD}, but oritatami systems have not been used to self-assemble a shape, to our knowledge. 

This paper proposes an oritatami system that self-assembles an $n$-bit fraction of the Heighway dragon (see Figure~\ref{} for 6-bit fraction), a kind of self-similar fractal dragon curve, for an arbitrary integer $n \ge 1$. 
The $n$-bit fraction consists of the first $2^n$ turns of the Heighway dragon. 
This fractal is also called the \textit{paperfolding sequence} named after its generating method: fold a paper slip in the middle, fold the result further in the middle, and so on. 
The result being unfolded consists of left and right turns as $P = LLRLLRRL \cdots$, which is the paperfolding sequence. 
There exists a deterministic finite automaton with output (DFAO) that returns the $i$-th element of $P$ when the binary representation of $i$ is fed into it from the least significant bit (LSB) \cite{AlloucheShallit2003}. 
We implement a binary counter, this DFAO, and a turning mechanism in oritatami systems that share a common interface through which they can interlock with each other and propagate a current count $i$. 
The turning mechanism is implemented by combining three instances of a module that bifurcates a given bit string in two ways and guides the system towards one of them according to the output of the preceding DFAO. 
Since the oritatami binary counter is an existing technology \cite{GeMeScSe2016}, we will explain the DFAO and turning mechanism in this paper. 

\bibliographystyle{splncs03}
\bibliography{dna23}

\end{document}


