\section*{Appendix}
\subsection{Turner}
Turner consists of two parts: bit string bifurcator and steering arm (colored in blue in Fig.~\ref{over_all}).
%steering arm
\begin{figure}[t]
 \centering
\scalebox{0.45}{
 \includegraphics{pic/overall_turn_part.pdf}
}
 \caption{The module-level abstraction of folding of Turner.
 All the white modules in the middle are spacers, some of which are implemented in the shape of parallelogram instead of glider. 
 %Note that the steering arm, colored in blue here, folds upward if AO = O or leftward otherwise, though both of them are drawn in here.
 }
\label{over_all}
\end{figure}

The bifurcator sends binary information as in Fig.~\ref{fig:abst_dragon} while folding into zigzags.
For that, it employs four types of module that propagates 1bit vertically (body-rpx, body-lpx1, and body-lpx2 in Fig.~\ref{over_all}), that lets 1bit cross another 1bit (body-gx), that forks 1bit vertically and horizontally (body-rgy, body-lgy), and that undergoes transition from a zig to a zag or from a zag to a zig and exposes 1bit outside (turn-rgp, turn-lgp).
The first two of them have already been implemented (see, e.g., \cite{HaKiOtSe2016}) so that we shall explain the others.

\begin{figure}[t]
 \centering
\scalebox{1}{
 \includegraphics[width=\linewidth]{pic/body_rlgy.pdf}
}
 \caption{(left) The possible two conformations of body-rgy and (right) of body-lgy.}
\label{body-y}
\end{figure}

The module body-rgy takes one of the two conformations in Fig.~\ref{body-y} (left) depending on the 1bit encoded in the two beads above.
Output below, the 1bit is encoded as a type of the second bead from left, while output right, it is encoded as whether this module ends top or bottom.
Its zag-variant, body-lgy, is implemented analogously; for its conformations, see Fig.~\ref{body-y} (right).

\begin{figure}[t]
 \centering
\scalebox{1}{
 \includegraphics{pic/body_lrgp.pdf}
}
 \caption{(top) The possible two conformations of turn-lgp and (bottom) of turn-rgp.}
\label{turn}
\end{figure}

The 1bit forked rightward by a body-rgy transfers till the end of the zig without being jammed because all remaining modules in the zig are designed in such a way that they start and end at the same height like the glider.
The module turn-rgp receives the 1bit (top or bottom), and exposes it by taking one of the two comformations in Fig.~\ref{turn} (bottom).
The module turn-lgp functions analogously in zags as being holded in Fig.~\ref{turn} (top).

\begin{figure}[t]
 \centering
\scalebox{1}{
 \includegraphics[width=\linewidth]{pic/change_route.pdf}
}
 \caption{The possible two conformations of change-route.}
\label{change_route}
\end{figure}

The bifurcator also propagates a signal, A or O, fed by the DFAO, to tell the steering arm which way it should take.
Specifically, the signal has the first module, change-route, of the steering arm take one of the two conformations in Fig.~\ref{change_route}, guiding the rest of the steering arm in the ordered direction.
The steering arm is provided with move modules, which let the bifurcated bit string pass through.  
Note that the Turner does not have to bifurcate AO.
Indeed, the second Turner is supposed to turn in the same manner as the first one.
It hence suffices to append A and O to the bifurcated bit strings on the acute side and obtuse side, respectively, as shown in Fig.~\ref{over_all}.


\begin{figure}[H]
  \begin{tabular}{c}
 
  
  \begin{minipage}{0.5\hsize}
  \centering
  \scalebox{0.5}{
  \begin{tikzpicture}[node distance=1cm,every node/.style={draw,circle,fill,inner sep=1pt}]
   \node[blue,inner sep = 2pt] (3) at (0:0)[label=above right:703]{};
  \node (2) at (120:1)[label=above right:702]{};
  \node (1)at (120:2)[label=above right:701]{};
  \node[right of= 1] (6) [label=above right:706]{};
  \node[right of =6](7)[label=above right:707]{};
  \node[right of =7](12)[label=above right:712]{};
\node[right of =12](13)[label=above right:713]{};
\node[right of =13](18)[label=above right:718]{};
  \node[right of =2](5)[label=above right:705]{};
  \node[right of =5](8)[label=above right:708]{};
  \node[right of =8](11)[label=above right:711]{};
\node[right of =11](14)[label=above right:714]{};
\node[right of =14](17)[label=above right:717]{};
  \node[right of =3,blue,inner sep = 2pt](4)[label=above right:704]{};
  \node[right of =4,blue,inner sep = 2pt](9)[label=above right:709]{};
  \node[right of =9,blue,inner sep = 2pt](10)[label=above right:710]{};
\node[right of =10](15)[label=above right:715]{};
\node[right of =15](16)[label=above right:716]{};
\node[above =1.6cm of  5,fill = green,inner sep = 2pt](100){};
  \draw[thick](1)--(2)--(3)--(4)--(5)--(6)--(7)--(8)--(9)--(10)--(11)--(12)--(13)--(14)--(15)--(16)--(17)--(18);
  \draw[dashed,thick,red](2)--(5);
  \draw[dashed,thick,red](5)--(8);
  \draw[dashed,thick,red](4)--(9);
  \draw[dashed,thick,red](5)--(7);
  \draw[dashed,thick,red](6)--(100);
  \draw[dashed,thick,red](7)--(100);
  \draw[dashed,thick,red](7)--(12);
  \draw[dashed,thick,red](8)--(11);
  \draw[dashed,thick,red](11)--(14);
  \draw[dashed,thick,red](10)--(15);
  \draw[dashed,thick,red](13)--(18);
  \draw[dashed,thick,red](15)--(17);
  \end{tikzpicture}
  }

  \end{minipage}

 \begin{minipage}{0.5\hsize}
  \centering
  \scalebox{0.5}{
  \begin{tikzpicture}[node distance=1cm,every node/.style={draw,circle,fill,inner sep=1pt}]
  \node [red,inner sep = 2pt] (11) at (0:0)[label=above right:711]{};
  \node (10) at (120:1)[label=above right:710]{};
  \node (1)at (120:2)[label=above right:701]{};
  \node[right of= 1] (2) [label=above right:702]{};
  \node[right of =2](3)[label=above right:703]{};
  \node[right of =3](4)[label=above right:704]{};
  \node[right of =4](5)[label=above right:705]{};
  \node[right of =5](18)[label=above right:718]{};
  \node[right of =10](9)[label=above right:709]{};
  \node[right of =9](8)[label=above right:708]{};
  \node[right of =8](7)[label=above right:707]{};
  \node[right of =7](6)[label=above right:706]{};
  \node[right of =6](17)[label=above right:717]{};
  \node[right of =11,red,inner sep = 2pt](12)[label=above right:712]{};
  \node[right of =12,red,inner sep = 2pt](13)[label=above right:713]{};
  \node[right of =13,red,inner sep = 2pt](14)[label=above right:714]{};
  \node[right of =14](15)[label=above right:715]{};
  \node[right of =15](16)[label=above right:716]{};
\node[above =1.6cm of 9,fill = green,inner sep = 2pt](100){};
  \draw[thick](1)--(2)--(3)--(4)--(5)--(6)--(7)--(8)--(9)--(10)--(11)--(12)--(13)--(14)--(15)--(16)--(17)--(18);
  \draw[dashed,thick,red](2)--(100);
  \draw[dashed,thick,red](2)--(100);
  \draw[dashed,thick,red](1)--(10);
  \draw[dashed,thick,red](2)--(10);
  \draw[dashed,thick,red](3)--(8);
  \draw[dashed,thick,red](4)--(7);
  \draw[dashed,thick,red](5)--(7);
  \draw[dashed,thick,red](7)--(13);
  \draw[dashed,thick,red](6)--(14);
  \draw[dashed,thick,red](6)--(15);
  \draw[dashed,thick,red](6)--(17);
  \draw[dashed,thick,red](15)--(17);
  \end{tikzpicture}
  }
 
  \end{minipage}

 \\
  
  \begin{minipage}{0.5\hsize}
  \centering
  \scalebox{0.5}{
  \begin{tikzpicture}[node distance=1cm,every node/.style={draw,circle,fill,inner sep=1pt}]
   \node[blue,inner sep = 2pt] (3) at (0:0)[label=above right:1803]{};
  \node (2) at (120:1)[label=above right:1802]{};
  \node (1)at (120:2)[label=above right:1801]{};
  \node[left of= 1] (6) [label=above right:1806]{};
  \node[left of =6](7)[label=above right:1807]{};
  \node[left of =7](12)[label=above right:1812]{};
\node[left of =12](13)[label=above right:1813]{};
\node[left of =13](18)[label=above right:1818]{};
  \node[left of =2](5)[label=above right:1805]{};
  \node[left of =5](8)[label=above right:1808]{};
  \node[left of =8](11)[label=above right:1811]{};
\node[left of =11](14)[label=above right:1814]{};
\node[left of =14](17)[label=above right:1817]{};
  \node[left of =3,blue,inner sep = 2pt](4)[label=above right:1804]{};
  \node[left of =4,blue,inner sep = 2pt](9)[label=above right:1809]{};
  \node[left of =9,blue,inner sep = 2pt](10)[label=above right:1810]{};
\node[left of =10](15)[label=above right:1815]{};
\node[left of =15](16)[label=above right:1816]{};
\node[above =1.6cm of  11,fill = green,inner sep = 2pt](100){};
\node[above =1.6cm of  14,fill = green,inner sep = 2pt](101){};
  \draw[thick](1)--(2)--(3)--(4)--(5)--(6)--(7)--(8)--(9)--(10)--(11)--(12)--(13)--(14)--(15)--(16)--(17)--(18);
  \draw[dashed,thick,red](2)--(5);
  \draw[dashed,thick,red](5)--(8);
  \draw[dashed,thick,red](8)--(11);
  \draw[dashed,thick,red](11)--(14);
  \draw[dashed,thick,red](14)--(17);
  \draw[dashed,thick,red](4)--(9);
  \draw[dashed,thick,red](7)--(12);
  \draw[dashed,thick,red](10)--(15);
  \draw[dashed,thick,red](7)--(100);
  \draw[dashed,thick,red](12)--(100);
  \end{tikzpicture}
  }
  \end{minipage}

 \begin{minipage}{0.5\hsize}
  \centering
  \scalebox{0.5}{
  \begin{tikzpicture}[node distance=1cm,every node/.style={draw,circle,fill,inner sep=1pt}]
  \node [red,inner sep = 2pt] (11) at (0:0)[label=above right:1811]{};
  \node (10) at (120:1)[label=above right:1810]{};
  \node (1)at (120:2)[label=above right:1801]{};
  \node[left of= 1] (2) [label=above right:1802]{};
  \node[left of =2](3)[label=above right:1803]{};
  \node[left of =3](4)[label=above right:1804]{};
  \node[left of =4](5)[label=above right:1805]{};
  \node[left of =5](18)[label=above right:1818]{};
  \node[left of =10](9)[label=above right:1809]{};
  \node[left of =9](8)[label=above right:1808]{};
  \node[left of =8](7)[label=above right:1807]{};
  \node[left of =7](6)[label=above right:1806]{};
  \node[left of =6](17)[label=above right:1817]{};
  \node[left of =11,red,inner sep = 2pt](12)[label=above right:1812]{};
  \node[left of =12,red,inner sep = 2pt](13)[label=above right:1813]{};
  \node[left of =13,red,inner sep = 2pt](14)[label=above right:1814]{};
  \node[left of =14](15)[label=above right:1815]{};
  \node[left of =15](16)[label=above right:1816]{};
\node[above =1.6cm of 6,fill = green,inner sep = 2pt](100){};
\node[above =1.6cm of 7,fill = green,inner sep = 2pt](101){};
  \draw[thick](1)--(2)--(3)--(4)--(5)--(6)--(7)--(8)--(9)--(10)--(11)--(12)--(13)--(14)--(15)--(16)--(17)--(18);
  \draw[dashed,thick,red](1)--(10);
  \draw[dashed,thick,red](2)--(8);
  \draw[dashed,thick,red](3)--(8);
  \draw[dashed,thick,red](9)--(13);
  \draw[dashed,thick,red](7)--(14);
  \draw[dashed,thick,red](6)--(15);
  \draw[dashed,thick,red](5)--(18);
  \draw[dashed,thick,red](100)--(5);
  \draw[dashed,thick,red](100)--(4);
  \draw[dashed,thick,red](101)--(4);
  \draw[dashed,thick,red](101)--(3);
  \end{tikzpicture}
  }
  \end{minipage}
 
\\

  \begin{minipage}{0.5\hsize}
  \centering
  \scalebox{0.5}{
  \begin{tikzpicture}[node distance=1cm,every node/.style={draw,circle,fill,inner sep=1pt}]
  \node (3) at (0:0) [blue,inner sep =2pt,label=above right:903]{};
  \node (2) at (120:1)[label=above right:902]{};
  \node (1)at (120:2)[label=above right:901]{};
  \node[left of= 1] (6) [label=above right:906]{};
  \node[left of =6](7)[label=above right:907]{};
  \node[left of =7](12)[label=above right:912]{};
  \node[left of =2](5)[label=above right:905]{};
  \node[left of =5](8)[label=above right:908]{};
  \node[left of =8](11)[label=above right:911]{};
  \node[left of =3](4)[blue,inner sep = 2pt,label=above right:904]{};
  \node[left of =4,](9)[label=above right:909]{};
  \node[left of =9](10)[label=above right:910]{};
  

\node[above =1.6cm of  8,fill = green,inner sep = 2pt](100){};
  \draw[thick](1)--(2)--(3)--(4)--(5)--(6)--(7)--(8)--(9)--(10)--(11)--(12);
  \draw[dashed,thick,red](2)--(5);
  \draw[dashed,thick,red](5)--(8);
  \draw[dashed,thick,red](8)--(11);
  \draw[dashed,thick,red](4)--(9);
  \end{tikzpicture}
  }
  
  \end{minipage}

 \begin{minipage}{0.5\hsize}
  \centering
  \scalebox{0.5}{
  \begin{tikzpicture}[node distance=1cm,every node/.style={draw,circle,fill,inner sep=1pt}]
  \node (7) at (0:0)[red,inner sep = 2pt,label=above right:907]{};
  \node (6) at (120:1)[label=above right:906]{};
  \node (1)at (120:2)[label=above right:901]{};
  \node[left of= 1] (2) [label=above right:902]{};
  \node[left of =2](3)[label=above right:903]{};
  \node[left of =3](12)[label=above right:912]{};
  \node[left of =6](5)[label=above right:905]{};
  \node[left of =7,red,inner sep = 2pt](8)[label=above right:908]{};
  \node[left of =8](9)[label=above right:909]{};
  \node[left of =9](10)[label=above right:910]{};
  \node[left of =5](4)[label=above right:904]{};
  \node[left of =4](11)[label=above right:911]{};
  
\node[above =1.6cm of 4,fill = green,inner sep = 2pt](100){};

  \draw[thick](1)--(2)--(3)--(4)--(5)--(6)--(7)--(8)--(9)--(10)--(11)--(12);
  \draw[dashed,thick,red](2)--(5);
  \draw[dashed,thick,red](5)--(8);
  \draw[dashed,thick,red](4)--(9);
  \draw[dashed,thick,red](4)--(10);
  \draw[dashed,thick,red](2)--(100);
  \draw[dashed,thick,red](3)--(100);
  \end{tikzpicture}
  }
  
  \end{minipage}
  \end{tabular}
  \caption{%body-lpx2
(top) The possible two conformations of body-rpx, (middle) The possible two conformations of body-lpx1, (bottom) The possible two conformations of body-lpx2.}
\label{body-px}
\end{figure} 


\begin{figure}[H]
  \begin{tabular}{c}
  \begin{minipage}{0.25\hsize}
  \centering
  \scalebox{0.5}{
  \begin{tikzpicture}[node distance=1cm,every node/.style={draw,circle,fill,inner sep=1pt}]
  \node (3) at (0:0)[label=above right:1006]{};
  \node (2) at (60:1)[label=above right:1005]{};
  \node (6)at (60:2)[label=above right:1009]{};
  \node[right of =3,red,inner sep = 2pt](4)[label=above right:1007]{};
  \node[right of =4](9)[label=above right:1012]{};
  \node[right of =9](10)[label=above right:1013]{};
  
  \node[right of =2](5)[label=above right:1008]{};
  \node[right of =5](8)[label=above right:1011]{};
  \node[right of =8](11)[label=above right:1014]{};
  
  \node[left of= 6] (1) [label=above right:1004]{};
  \node[right of =6](7)[label=above right:1010]{};
  \node[right of =7](12)[label=above right:1015]{};
  \node[above =1.6cm of 8,fill = green,inner sep = 2pt](100){};
  \node[above =1.6cm of 5,fill = green,inner sep = 2pt](101){};
  \draw[thick](1)--(2)--(3)--(4)--(5)--(6)--(7)--(8)--(9)--(10)--(11)--(12);
  \draw[dashed,thick,red](1)--(6);
  \draw[dashed,thick,red](2)--(5);
  \draw[dashed,thick,red](4)--(9);
  \draw[dashed,thick,red](5)--(8);
  \draw[dashed,thick,red](7)--(12);
  \draw[dashed,thick,red](12)--(100);
  \end{tikzpicture}
  }
  
  \end{minipage}

\begin{minipage}{0.25\hsize}
  \centering
  \scalebox{0.5}{
  \begin{tikzpicture}[node distance=1cm,every node/.style={draw,circle,fill,inner sep=1pt}]
  \node (1) at (0:0)[label=above right:1004]{};
  \node (2) at (60:1)[label=above right:1005]{};
  \node (4)at (60:2)[label=above right:1007]{};
  \node[left of= 4] (3) [label=above right:1006]{};
  \node[right of =4](5)[label=above right:1008]{};
    \node[right of =5](6)[label=above right:1009]{};
    \node[right of =2](9)[label=above right:1012]{};
    \node[right of =9](8)[label=above right:1011]{};
    \node[right of =8](7)[label=above right:1010]{};
    \node[right of =1,red,inner sep = 2pt](10)[label=above right:1013]{};
    \node[right of =10](11)[label=above right:1014]{};
    \node[right of =11](12)[label=above right:1015]{};
  
  \node[above =1.6cm of 8,fill = green,inner sep = 2pt](100){};
\node[above =1.6cm of 9,fill = green,inner sep = 2pt](101){};
  \draw[thick](1)--(2)--(3)--(4)--(5)--(6)--(7)--(8)--(9)--(10)--(11)--(12);
  \draw[dashed,thick,red](4)--(9);
  \draw[dashed,thick,red](5)--(8);
  \draw[dashed,thick,red](7)--(12);
  \draw[dashed,thick,red](1)--(10);
  \draw[dashed,thick,red](5)--(100);
  \draw[dashed,thick,red](6)--(100);
  \end{tikzpicture}
  }
 
  \end{minipage}

\begin{minipage}{0.25\hsize}
  \centering
  \scalebox{0.5}{
  \begin{tikzpicture}[node distance=1cm,every node/.style={draw,circle,fill,inner sep=1pt}]
  \node (1) at (0:0)[label=above right:1004]{};
  \node (6) at (-60:1)[label=above right:1009]{};
  \node (8)at (-60:2)[blue,inner sep = 2pt,label=above right:1011]{};
 \node[left of= 8] (7) [label=above right:1010]{};
  \node[right of =1](2)[label=above right:1005]{};
  \node[right of =2](3)[label=above right:1006]{};
  \node[right of =6](5)[label=above right:1008]{};
  \node[right of =5](4)[label=above right:1007]{};
  \node[right of =8](9)[label=above right:1012]{};
  \node[right of =9](10)[label=above right:1013]{};
  \node[right of =3](12)[label=above right:1015]{};
  \node[right of =4](11)[label=above right:1014]{};
  
 
  \node[above =1.6cm of 4,fill = green,inner sep = 2pt](100){};
\node[above =1.6cm of 5,fill = green,inner sep = 2pt](101){};
  
  \draw[thick](1)--(2)--(3)--(4)--(5)--(6)--(7)--(8)--(9)--(10)--(11)--(12);
  \draw[dashed,thick,red](1)--(6);
  \draw[dashed,thick,red](2)--(5);
  \draw[dashed,thick,red](5)--(8);
  \draw[dashed,thick,red](4)--(9);
  \draw[dashed,thick,red](3)--(12);
  \draw[dashed,thick,red](2)--(101);
  \draw[dashed,thick,red](3)--(101);
  \draw[dashed,thick,red](3)--(100);
  \draw[dashed,thick,red](12)--(100);
  \end{tikzpicture}
  }
  
  \end{minipage}


\begin{minipage}{0.25\hsize}
  \centering
  \scalebox{0.5}{
  \begin{tikzpicture}[node distance=1cm,every node/.style={draw,circle,fill,inner sep=1pt}]
  \node (3) at (0:0)[label=above right:1006]{};
  \node (2) at (-60:1)[label=above right:1005]{};
  \node (6)at (-60:2)[blue,inner sep = 2pt,label=above right:1009]{};
  \node[right of =3](4)[label=above right:1007]{};
  \node[right of =4](9)[label=above right:1012]{};
  \node[right of =9](10)[label=above right:1013]{};
  
  \node[right of =2](5)[label=above right:1008]{};
  \node[right of =5](8)[label=above right:1011]{};
  \node[right of =8](11)[label=above right:1014]{};
  
  \node[left of= 6] (1) [label=above right:1004]{};
  \node[right of =6](7)[label=above right:1010]{};
  \node[right of =7](12)[label=above right:1015]{};

  \node[above =1.6cm of 8,fill = green,inner sep = 2pt](){};
\node[above =1.6cm of 5,fill = green,inner sep = 2pt](){};
  \draw[thick](1)--(2)--(3)--(4)--(5)--(6)--(7)--(8)--(9)--(10)--(11)--(12);
  \draw[dashed,thick,red](1)--(6);
  \draw[dashed,thick,red](7)--(12);
  \draw[dashed,thick,red](2)--(5);
  \draw[dashed,thick,red](5)--(8);
  \draw[dashed,thick,red](4)--(9);
  \end{tikzpicture}
  }
 
  \end{minipage}
  \end{tabular}
  \caption{The possible four conformations of body-gx.}
\label{body-gx}
\end{figure} 



\begin{figure}[H]
  \begin{tabular}{c}
 
  
  \begin{minipage}{0.5\hsize}
  \centering
  \scalebox{0.5}{
  \begin{tikzpicture}[node distance=1cm,every node/.style={draw,circle,fill,inner sep=1pt}]
   \node (1) at (0:0)[label=above right:1551]{};
  \node (2) at (120:1)[label=above right:1552]{};
  \node (3)at (120:2)[label=above right:1553]{};
  \node[left of= 1] (6) [label=above right:1556]{};
  \node[left of =6](7)[label=above right:1557]{};
  \node[left of =7,blue,inner sep = 2pt](12)[label=above right:1562]{};
\node[left of =12](13)[label=above right:1563]{};
\node[left of =13](18)[label=above right:1568]{};
  \node[left of =2](5)[label=above right:1555]{};
  \node[left of =5](8)[label=above right:1558]{};
  \node[left of =8](11)[label=above right:1561]{};
\node[left of =11](14)[label=above right:1564]{};
\node[left of =14](17)[label=above right:1567]{};
  \node[left of =3](4)[label=above right:1554]{};
  \node[left of =4](9)[label=above right:1559]{};
  \node[left of =9](10)[label=above right:1560]{};
\node[left of =10](15)[label=above right:1565]{};
\node[left of =15](16)[label=above right:1566]{};
\coordinate[left of = 17](99){};
\node[above =1.6cm of  17,fill = green,inner sep = 2pt](100){};
\node[above =1.6cm of  99,fill = green,inner sep = 2pt](101){};
  \draw[thick](1)--(2)--(3)--(4)--(5)--(6)--(7)--(8)--(9)--(10)--(11)--(12)--(13)--(14)--(15)--(16)--(17)--(18);
  \draw[dashed,thick,red](1)--(6);
  \draw[dashed,thick,red](2)--(6);
\draw[dashed,thick,red](7)--(12);
\draw[dashed,thick,red](13)--(18);
\draw[dashed,thick,red](18)--(14);
\draw[dashed,thick,red](14)--(17);
\draw[dashed,thick,red](16)--(101);
\draw[dashed,thick,red](15)--(100);
  \end{tikzpicture}
  }
  \end{minipage}

 \begin{minipage}{0.5\hsize}
  \centering
  \scalebox{0.5}{
  \begin{tikzpicture}[node distance=1cm,every node/.style={draw,circle,fill,inner sep=1pt}]
   \node (1) at (0:0)[label=above right:1551]{};
  \node (2) at (120:1)[label=above right:1552]{};
  \node (3)at (120:2)[label=above right:1553]{};
  \node[left of= 1] (6) [label=above right:1556]{};
  \node[left of =6](7)[label=above right:1557]{};
  \node[left of =7,red,inner sep = 2pt](16)[label=above right:1566]{};
\node[left of =16](17)[label=above right:1567]{};
\node[left of =17](18)[label=above right:1568]{};
  \node[left of =2](5)[label=above right:1555]{};
  \node[left of =5](8)[label=above right:1558]{};
  \node[left of =8](15)[label=above right:1565]{};
\node[left of =15](14)[label=above right:1564]{};
\node[left of =14](13)[label=above right:1563]{};
  \node[left of =3](4)[label=above right:1554]{};
  \node[left of =4](9)[label=above right:1559]{};
  \node[left of =9](10)[label=above right:1560]{};
\node[left of =10](11)[label=above right:1561]{};
\node[left of =11](12)[label=above right:1562]{};
\coordinate[left of = 13](99){};
\node[above =1.6cm of  13,fill = yellow,inner sep = 2pt](100){};
\node[above =1.6cm of  99,fill = yellow,inner sep = 2pt](101){};
  \draw[thick](1)--(2)--(3)--(4)--(5)--(6)--(7)--(8)--(9)--(10)--(11)--(12)--(13)--(14)--(15)--(16)--(17)--(18);
  \draw[dashed,thick,red](1)--(6);
\draw[dashed,thick,red](2)--(6);
\draw[dashed,thick,red](8)--(15);
\draw[dashed,thick,red](7)--(16);
\draw[dashed,thick,red](13)--(18);
\draw[dashed,thick,red](18)--(14);
\draw[dashed,thick,red](14)--(17);
\draw[dashed,thick,red](12)--(101);
\draw[dashed,thick,red](11)--(100);
  \end{tikzpicture}
  }
  \end{minipage}
 
 \end{tabular}
  \caption{The possible two conformations of move.}
\label{move}
\end{figure} 
\begin{remark}
In fact, as suggested in Fig.~\ref{over_all}, the bifurcator outputs the bit string also downward.
That is, it bifurcates a given bit string into three directions.
This provides a more space-efficient way to replicate a bit string many-folds.
\end{remark}
