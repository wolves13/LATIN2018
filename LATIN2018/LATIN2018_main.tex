\documentclass{llncs}
\usepackage[dvipdfmx]{graphicx}

\title{Bit bifurcation by cotranscriptional folding\thanks{This work is in part supported by JST Program to Disseminate Tenure Tracking System, MEXT, Japan, No. 6F36 and JSPS Grant-in-Aid for Young Scientists (A) No. 16H05854 to S.~S.}}
\author{Yusei Masuda \and Shinnosuke Seki\thanks{Corresponding author} \and Yuki Ubukata}
\institute{
Department of Computer and Network Engineering, 
The University of Electro-Communications, 
1-5-1, Chofugaoka, Chofu, Tokyo, 1828585, Japan \email{s.seki@uec.ac.jp}
}

\begin{document}

\maketitle

\begin{abstract}
We demonstrate cotranscriptional folding of a finite part of the Heighway dragon, a fractal also-known as the paperfolding sequence $P = LLRLLRRL \cdots$ by an oritatami system. 
The $i$-th element of $P$ can be obtained by feeding $i$ in binary to a 4-state deterministic finite automaton with output (DFAO). 
We implement this DFAO and a bit-string bifurcator as modules of oritatami system. 
Combining them with a known binary counter module yields the target oritatami system. 
%
\begin{keywords}
Heighway dragon, 
Fractal, 
Cotranscriptional folding, 
Oritatami system, 
Automatic sequence, 
Bitstring bifurcation
\end{keywords}
\end{abstract}

%%%%%%%%%%%%%%%%%%%%%%%%%%%%%%%%%%%%%%%%%%%%%%%%%%%%%%%%%%%%%%%%%%%%%
\subsection{Verificaion}
%%%%%%%%%%%%%%%%%%%%%%%%%%%%%%%%%%%%%%%%%%%%%%%%%%%%%%%%%%%%%%%%%%%%%

We explain the correctness of the folding.
The folding of each module depends on its environment, i.e. on the components already folded nearby and on the current minimum energy conformations output by previous step.
Algorithm~\ref{alg:dragon} is an algorithm of designing the $n$-th iteration of the Heighway dragon.
It shows all possible environments of each module and component.
Using the simulator developed for \cite{HaKiOtSe2016}, we have verified that all of the components fold correctly in all possible environments.


\begin{algorithm}                      
\caption{The $n$-th iteration of the Heighway dragon}         
\label{alg:dragon}                          
\begin{algorithmic}                  
\REQUIRE $n \geq , length \leq 0$

\STATE $current \Leftarrow 0$
\WHILE{$current < 2^n - 1$}

%%%%%%%%%%%%%%%%%%%Counter-zig%%%%%%%%%%%%%%%%%%%
\STATE $carry = 1$
\FOR{$i = 1$ to $n$}
\IF{$carry = 1$}

\IF{above component is body-rpx-0}
\STATE Cozig-11 is folded.
\STATE $carry \Leftarrow 0$ 
\ELSIF{above component is body-rpx-1}
\STATE Cozig-10 is folded.
\ENDIF

\ELSIF{$carry = 0$}

\IF{above component is body-rpx-0}
\STATE Cozig-00 is folded.
\ELSIF{above component is body-rpx-1}
\STATE Cozig-01 is folded.
\ENDIF
\ENDIF
\STATE Spacer is folded.
\ENDFOR

\IF{$carry =1$}
\STATE the folding stops (counter capacity exceeded)
\ENDIF
%%%%%%%%%%%%%%%%%%%Counter-zig%%%%%%%%%%%%%%%%%%%
 
\STATE $current \Leftarrow current + 1$
\ENDWHILE

\end{algorithmic}
\end{algorithm}



\bibliographystyle{splncs03}
\bibliography{dna23}

\end{document}


