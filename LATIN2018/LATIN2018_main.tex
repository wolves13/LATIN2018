\documentclass{llncs}

\title{Bit bifurcation by cotranscriptional folding\thanks{This work is in part supported by JST Program to Disseminate Tenure Tracking System, MEXT, Japan, No. 6F36 and JSPS Grant-in-Aid for Young Scientists (A) No. 16H05854 to S.~S.}}
\author{Yusei Masuda \and Shinnosuke Seki\thanks{Corresponding author} \and Yuki Ubukata}
\institute{
Department of Computer and Network Engineering, 
The University of Electro-Communications, 
1-5-1, Chofugaoka, Chofu, Tokyo, 1828585, Japan \email{s.seki@uec.ac.jp}
}

\begin{document}

\maketitle

\begin{abstract}
We demonstrate cotranscriptional folding of a finite part of the Heighway dragon, a fractal also-known as the paperfolding sequence $P = LLRLLRRL \cdots$ by an oritatami system. 
The $i$-th element of $P$ can be obtained by feeding $i$ in binary to a 4-state deterministic finite automaton with output (DFAO). 
We implement this DFAO and a bit-string bifurcator as modules of oritatami system. 
Combining them with a known binary counter module yields the target oritatami system. 
%
\begin{keywords}
Heighway dragon, 
Fractal, 
Cotranscriptional folding, 
Oritatami system, 
Automatic sequence, 
Bitstring bifurcation
\end{keywords}
\end{abstract}

\bibliographystyle{splncs03}
\bibliography{dna23}

\end{document}


