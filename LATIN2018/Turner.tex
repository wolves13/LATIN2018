%-------------------------------------------------------------------------------------------
			\subsubsection{Turning module}
%-------------------------------------------------------------------------------------------

\begin{figure}[t]
\centering
\includegraphics[width=\linewidth]{pic/overall_turn_part.pdf}
\caption{
Component-level abstraction of folding of turning module.
All the white components in the middle are spacers, some of which are implemented in the shape of parallelogram instead of glider. 
 }
\label{fig:overall_turning}
\end{figure}

Lastly, we explain the turning module. 
It consists of two parts: bit-sequence bifurcation submodule and steering arm component (colored in blue in Figure~\ref{fig:overall_turning}). 

\begin{figure}[h]
\centering
\includegraphics[width=0.8\linewidth]{pic/body-rpx.pdf}
\caption{The possible two conformations of body-rpx component.}
\label{fig:body-rpx}
\end{figure}

\begin{figure}[h]
\centering
\includegraphics[width=0.8\linewidth]{pic/body-lpx1.pdf}
\caption{The possible two conformations of body-lpx1 component.}
\label{fig:body-lpx1}
\end{figure}

\begin{figure}[h]
\centering
\includegraphics[width=0.55\linewidth]{pic/body-lpx2.pdf}
\caption{The possible two conformations of body-lpx2 component.}
\label{fig:body-lpx2}
\end{figure}

\begin{figure}[h]
\centering
\includegraphics[width=\linewidth]{pic/body-gx.pdf}
\caption{The possible four conformations of body-gx component.}
\label{fig:body-gx}
\end{figure}

\begin{figure}[h]
\centering
\includegraphics[width=0.5\linewidth]{pic/body-rgy.pdf}
\caption{The possible two conformations of body-rgy component.}
\label{fig:body-rgy}
\end{figure}

\begin{figure}[h]
\centering
\includegraphics[width=0.5\linewidth]{pic/body-lgy.pdf}
\caption{The possible two conformations of body-lgy component.}
\label{fig:body-lgy}
\end{figure}

\begin{figure}[h]
\centering
\includegraphics[width=0.7\linewidth]{pic/turn-rgp.pdf}
\caption{The possible two conformations of turn-rgp component.}
\label{fig:turn-rgp}
\end{figure}

\begin{figure}[h]
\centering
\includegraphics[width=0.8\linewidth]{pic/turn-lgp.pdf}
\caption{The possible two conformations of turn-lgp component.}
\label{fig:turn-lgp}
\end{figure}



The bifurcation submodule sends bits of the current count $i$ and the reinterpreted signal (A or O) as shown in Figure~\ref{fig:abst_dragon} while folding into zig-zags.
For that, it employs the following four types of components: 
\begin{enumerate}[itemsep=0pt]
\item components to propagate 1-bit vertically: body-rpx (Figure~\ref{fig:body-rpx}), body-lpx1 (Figure~\ref{fig:body-lpx1}), body-lpx2 (Figure~\ref{fig:body-lpx2}); 
\item a component to let 1-bit cross another 1-bit: body-gx (Figure~\ref{fig:body-gx}); 
\item components to fork 1-bit vertically and horizontally: body-rgy (Figure~\ref{fig:body-rgy}) and body-lgy (Figure~\ref{fig:body-lgy});  
\item components to undergo transition between a zig and a zag and exposes 1-bit outside: turn-rgp (Figure~\ref{fig:turn-rgp}) and turn-lgp (Figure~\ref{fig:turn-lgp}). 
\end{enumerate} 
%that propagates 1bit vertically (body-rpx, body-lpx1, and body-lpx2 in Figure~\ref{fig:overall_turning}), that lets 1bit cross another 1bit (body-gx), that forks 1bit vertically and horizontally (body-rgy, body-lgy), and that undergoes transition from a zig to a zag or from a zag to a zig and exposes 1bit outside (turn-rgp, turn-lgp).
The first two types have already been implemented (see, e.g., \cite{HaKiOtSe2016}) so that we shall explain the others.

The component body-rgy takes one of the two conformations in Figure~\ref{fig:body-rgy} depending on the 1-bit encoded in the two beads above.
Output below, the 1-bit is encoded as a type of the second bead from left, while output right, it is encoded as the position of its last bead (top or bottom).
Its zag-variant, body-lgy, is implemented analogously; for its conformations.

The 1bit forked rightward by a body-rgy transfers till the end of the zig without being jammed because all remaining modules in the zig are designed in such a way that they start and end at the same height like the even-distance glider.
The module turn-rgp receives the 1bit (top or bottom), and exposes it by taking one of the two comformations in Figure~\ref{fig:turn-rgp}.
The module turn-lgp functions analogously in zags as being folded in Figure~\ref{fig:turn-lgp}.

\begin{figure}[h]
\centering
\includegraphics[width=\linewidth]{pic/change_route.pdf}
\caption{The possible two conformations of change-route component.}
\label{fig:change_route}
\end{figure}

\begin{figure}[h]
\centering
\includegraphics[width=\linewidth]{pic/move.pdf}
\caption{The possible two conformations of move component.}
\label{fig:move}
\end{figure}

The bifurcation submodule also propagates the 1-bit A or O, output by the DFAO, to tell the steering arm which way it should take.
Specifically, the signal has the first module, change-route, of the steering arm take one of the two conformations in Figure~\ref{fig:change_route}, guiding the rest of the steering arm towards the ordered direction.
The steering arm is provided with move modules (Figure~\ref{fig:move}), which let bits of the bifurcated bit sequence pass through.  
Note that the turning module does not have to bifurcate AO.
Indeed, the second and third turning modules are supposed to turn in the same manner as the first one.
It hence suffices to append A and O to the bifurcated bit sequences on the acute side and obtuse side, respectively, as shown in Figure~\ref{fig:overall_turning}.

\begin{remark}
In fact, as suggested in Figure~\ref{fig:overall_turning}, the bifurcation component outputs an input bit sequence also downward.
That is, it bifurcates the sequence into three ways.
This provides a more space-efficient way to replicate a bit sequence many-folds.
\end{remark}

